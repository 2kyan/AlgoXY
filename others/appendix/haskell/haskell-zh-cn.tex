\ifx\wholebook\relax \else
% ------------------------

\documentclass[UTF8]{article}
%------------------- Other types of document example ------------------------
%
%\documentclass[twocolumn]{IEEEtran-new}
%\documentclass[12pt,twoside,draft]{IEEEtran}
%\documentstyle[9pt,twocolumn,technote,twoside]{IEEEtran}
%
%-----------------------------------------------------------------------------
%\input{../../../common.tex}
%
% loading packages
%

\RequirePackage{ifpdf}
\RequirePackage{ifxetex}

%
%
\ifpdf
  \RequirePackage[pdftex,%
       bookmarksnumbered,%
              colorlinks,%
          linkcolor=blue,%
              hyperindex,%
        plainpages=false,%
       pdfstartview=FitH]{hyperref}
\else\ifxetex
  \RequirePackage[bookmarksnumbered,%
               colorlinks,%
           linkcolor=blue,%
               hyperindex,%
         plainpages=false,%
        pdfstartview=FitH]{hyperref}
\else
  \RequirePackage[dvipdfm,%
        bookmarksnumbered,%
               colorlinks,%
           linkcolor=blue,%
               hyperindex,%
         plainpages=false,%
        pdfstartview=FitH]{hyperref}
\fi\fi
%\usepackage{hyperref}

% other packages
%--------------------------------------------------------------------------
\usepackage{graphicx, color}
\usepackage{subfig}
\usepackage{multicol}
\usepackage{tikz}
\usetikzlibrary{matrix,positioning,shapes}

\usepackage{amsmath, amsthm, amssymb} % for math
\usepackage{exercise} % for exercise
\usepackage{import} % for nested input

%
% for programming
%
\usepackage{verbatim}
\usepackage{fancyvrb}
\usepackage{listings}
%\usepackage{algorithmic} %old version; we can use algorithmicx instead
%\usepackage[plain]{algorithm} %remove rule (horizontal line on top/below the algorithm
\usepackage{algorithm} %to remove rules change to \usepackage[plain]{algorithm}
%\usepackage{algorithm2e}
\usepackage[noend]{algpseudocode} %for pseudo code, include algorithmicsx automatically
\usepackage{appendix}
\usepackage{makeidx} % for index support
\usepackage{titlesec}

\usepackage{fontspec}
\usepackage{xunicode}
%\usepackage{fontenc}
\usepackage{textcomp}
\usepackage{url}

% detect and select Chinese font
% ------------------------------
% fc-list :lang=zh    % list all Chinese fonts
% fc-list :mono       % list all mono fonts
% fc-cache            % refresh cache to load new installed fonts
\def\macmainfont{STSong}  % Under Mac OS X
\def\macboldfont{STHeiti}
\def\macmonofont{Monaco}
\def\winmainfont{SimSun} % Under Windows
\def\winboldfont{SimHei}
\def\winmonofont{Consolas}
\def\linuxmainfont{WenQuanYi Micro Hei} % Under Linux
\def\linuxboldfont{WenQuanYi Zen Hei}
\def\linuxmainfont{Courier}

\suppressfontnotfounderror1 % Avoid setting exit code (error level) to break make process
\count255=\interactionmode
\batchmode

% zh-cn main font
\let\cnmainft=\macmainfont
\font\thefont="\cnmainft"\space at 10pt
\ifx\thefont\nullfont
  \let\cnmainft=\winmainfont
  \font\thefont="\cnmainft"\space at 10pt
  \ifx\the\nullfont
    \let\cnmainft=\linuxmainfont
    \font\thefont="\cnmainft"\space at 10pt
    \ifx\the\nullfont
      \errorstopmode
      \errmessage{no suitable Chinese main font found}
    \fi
  \fi
\fi

% zh-cn bold font
\let\cnboldft=\macboldfont
\font\thefont="\cnboldft"\space at 10pt
\ifx\thefont\nullfont
  \let\cnboldft=\winboldfont
  \font\thefont="\cnboldft"\space at 10pt
  \ifx\the\nullfont
    \let\cnboldft=\linuxboldfont
    \font\thefont="\cnboldft"\space at 10pt
    \ifx\the\nullfont
      \errorstopmode
      \errmessage{no suitable Chinese bold font found}
    \fi
  \fi
\fi

% mono font
\let\monoft=\macmonofont
\font\thefont="\monoft"\space at 10pt
\ifx\thefont\nullfont
  \let\monoft=\winmonofont
  \font\thefont="\monoft"\space at 10pt
  \ifx\the\nullfont
    \let\monoft=\linuxmonofont
    \font\thefont="\monoft"\space at 10pt
    \ifx\the\nullfont
      \errorstopmode
      \errmessage{no suitable mono font found}
    \fi
  \fi
\fi

\interactionmode=\count255

\setmainfont[Mapping=tex-text, BoldFont=\cnboldft]{\cnmainft} % 正文字体(衬线字体)对应Roman
\setsansfont[BoldFont=\cnboldft]{\cnmainft}   % 无衬线字体
\setmonofont[Scale=MatchLowercase]{\monoft}   % 等宽字体

\XeTeXlinebreaklocale "zh"  % to solve the line breaking issue
\XeTeXlinebreakskip = 0pt plus 1pt minus 0.1pt

\titleformat{\paragraph}
{\normalfont\normalsize\bfseries}{\theparagraph}{1em}{}
\titlespacing*{\paragraph}
{0pt}{3.25ex plus 1ex minus .2ex}{1.5ex plus .2ex}

\lstdefinelanguage{Smalltalk}{
  morekeywords={self,super,true,false,nil,thisContext}, % This is overkill
  morestring=[d]',
  morecomment=[s]{"}{"},
  alsoletter={\#:},
  escapechar={!},
  literate=
    {BANG}{!}1
    {UNDERSCORE}{\_}1
    {\\st}{Smalltalk}9 % convenience -- in case \st occurs in code
    % {'}{{\textquotesingle}}1 % replaced by upquote=true in \lstset
    {_}{{$\leftarrow$}}1
    {>>>}{{\sep}}1
    {^}{{$\uparrow$}}1
    {~}{{$\sim$}}1
    {-}{{\sf -\hspace{-0.13em}-}}1  % the goal is to make - the same width as +
    %{+}{\raisebox{0.08ex}{+}}1		% and to raise + off the baseline to match -
    {-->}{{\quad$\longrightarrow$\quad}}3
	, % Don't forget the comma at the end!
  tabsize=2
}[keywords,comments,strings]

% for literate Haskell code
\lstdefinestyle{Haskell}{
  flexiblecolumns=false,
  basewidth={0.5em,0.45em},
  morecomment=[l]--,
  literate={+}{{$+$}}1 {/}{{$/$}}1 {*}{{$*$}}1 {=}{{$=$}}1
           {>}{{$>$}}1 {<}{{$<$}}1 {\\}{{$\lambda$}}1
           {\\\\}{{\char`\\\char`\\}}1
           {->}{{$\rightarrow$}}2 {>=}{{$\geq$}}2 {<-}{{$\leftarrow$}}2
           {<=}{{$\leq$}}2 {=>}{{$\Rightarrow$}}2
           {\ .}{{$\circ$}}2 {\ .\ }{{$\circ$}}2
           {>>}{{>>}}2 {>>=}{{>>=}}2
           {|}{{$\mid$}}1
}

\lstloadlanguages{C, C++, Lisp, Haskell, Python, Smalltalk}

\lstset{
  basicstyle=\small\ttfamily,
  commentstyle=\rmfamily,
  texcl=true,
  showstringspaces = false,
  upquote=true,
  flexiblecolumns=false
}

% ======================================================================

\def\BibTeX{{\rm B\kern-.05em{\sc i\kern-.025em b}\kern-.08em
    T\kern-.1667em\lower.7ex\hbox{E}\kern-.125emX}}

%
% mathematics
%
\newcommand{\be}{\begin{equation}}
\newcommand{\ee}{\end{equation}}
\newcommand{\bmat}[1]{\left( \begin{array}{#1} }
\newcommand{\emat}{\end{array} \right) }
\newcommand{\VEC}[1]{\mbox{\boldmath $#1$}}

% numbered equation array
\newcommand{\bea}{\begin{eqnarray}}
\newcommand{\eea}{\end{eqnarray}}

% equation array not numbered
\newcommand{\bean}{\begin{eqnarray*}}
\newcommand{\eean}{\end{eqnarray*}}

\newtheorem{theorem}{定理}[section]
\newtheorem{lemma}[theorem]{引理}
\newtheorem{proposition}[theorem]{Proposition}
\newtheorem{corollary}[theorem]{Corollary}

% 中文书籍设置
% ====================================
\renewcommand\contentsname{目\ 录}
%\renewcommand\listfigurename{插图目录}
%\renewcommand\listtablename{表格目录}
\renewcommand\figurename{图}
\renewcommand\tablename{表}
\renewcommand\proofname{证明}
\renewcommand\ExerciseName{练习}
%\renewcommand{\algorithmcfname}{算法}

\ifx\wholebook\relax
\renewcommand\bibname{参\ 考\ 文\ 献}                    %book类型
%\newtheorem{Definition}[Theorem]{定义}
\newtheorem{Theorem}{定理}[chapter]
\newtheorem{example}{例题}[chapter]
\else
\renewcommand\refname{参\ 考\ 文\ 献}
\fi

%\setcounter{secnumdepth}{4}
\titleformat{\chapter}
  {\normalfont\bfseries\Large}
  {第\arabic{chapter}章}
  {12pt}{\Large}
%% \titleformat{\subsection}
%%   {\normalfont\bfseries\large}
%%   {\CJKnumber{\arabic{subsection}}、}
%%   {12pt}{\large}
%% \titleformat{\subsubsection}
%%   {\normalfont\bfseries\normalsize}
%%   {\arabic{subsubsection}.}
%%   {12pt}{\normalsize}

%\renewcommand{\baselinestretch}{1.5}                        %文章行间距为1.5倍。

\makeatletter
\newcommand{\verbatimfont}[1]{\renewcommand{\verbatim@font}{\ttfamily#1}}
\makeatother

\setcounter{tocdepth}{4}
\setcounter{secnumdepth}{4}

%\verbatimfont{\footnotesize}


\setcounter{page}{1}

\begin{document}

%--------------------------

% ================================================================
%                 COVER PAGE
% ================================================================

\title{Haskell例子程序}

\author{Larry~LIU~Xinyu
\thanks{{\bfseries Larry LIU Xinyu } \newline
  Email: liuxinyu95@gmail.com \newline}
  }

\maketitle
\fi

\markboth{Haskell例子程序}{Elementary Algorithms}

\ifx\wholebook\relax
\chapter{Haskell例子程序}
\numberwithin{Exercise}{chapter}
\fi

本附录列出了各个章节中对应的Haskell例子程序。

\subsection{最小可用数字问题}
分而治之的最小可用数字问题解法。

\lstset{language=Haskell, frame=single}
\begin{lstlisting}
import Data.List

minFree xs = bsearch xs 0 (length xs - 1)

bsearch xs l u | xs == [] = l
               | length as == m - l + 1 = bsearch bs (m + 1) u
               | otherwise = bsearch as l m
    where
      m = (l + u) `div` 2
      (as, bs) = partition (<=m) xs
\end{lstlisting}

\subsection{正规数(哈明数)问题}
寻找只包含2、3、或5为因子的第$k$个数。下面的方法使用了Haskell的惰性求值特性和无穷流。

\begin{lstlisting}
merge (x:xs) (y:ys) | x <y = x : merge xs (y:ys)
                    | x == y = x : merge xs ys
                    | otherwise = y : merge (x:xs) ys

ns = 1:merge (map (*2) ns) (merge (map (*3) ns) (map (*5) ns))
\end{lstlisting}

下面的代码给出第1500个正规数。
\begin{verbatim}
last $ take 1500 ns
\end{verbatim}

我们也可以使用三个队列来产生正规数。

\begin{lstlisting}
ks 1 xs _ = xs
ks n xs (q2, q3, q5) = ks (n-1) (xs++[x]) update
    where
      x = minimum $ map head [q2, q3, q5]
      update | x == head q2 = ((tail q2)++[x*2], q3++[x*3], q5++[x*5])
             | x == head q3 = (q2, (tail q3)++[x*3], q5++[x*5])
             | otherwise = (q2, q3, (tail q5)++[x*5])

takeN n = ks n [1] ([2], [3], [5])
\end{lstlisting}

执行\texttt{last \$ takeN 1500}即产生第1500个正规数859963392。

\section{二叉搜索树}

\subsection{二叉搜索树的定义}

二叉搜索树的代数数据类型定义。

\begin{lstlisting}
data Tree a = Empty
            | Node (Tree a) a (Tree a)
\end{lstlisting}

\subsection{插入}

\begin{lstlisting}
insert Empty k = Node Empty k Empty
insert (Node l x r) k | k < x = Node (insert l k) x r
                      | otherwise = Node l x (insert r k)
\end{lstlisting}

\subsection{将二叉搜索树转换为有序列表}

\begin{lstlisting}
toList Empty = []
toList (Node l x r) = toList l ++ [x] ++ toList r
\end{lstlisting}

\subsection{二叉搜索树的查找}

\begin{lstlisting}
lookup Empty _ = Empty
lookup t@(Node l k r) x | k == x = t
                        | x < k = lookup l x
                        | otherwise = lookup r x
\end{lstlisting}

\subsection{从二叉搜索树中删除元素}

\begin{lstlisting}
delete Empty _ = Empty
delete (Node l k r) x | x < k = (Node (delete l x) k r)
                      | x > k = (Node l k (delete r x))
                      -- x == k
                      | isEmpty l = r
                      | isEmpty r = l
                      | otherwise = (Node l k' (delete r k'))
                          where k' = min r
\end{lstlisting}

\section{红黑树}

\subsection{插入}

使用模式匹配的红黑树插入算法。

\begin{lstlisting}
data Color = R | B
data RBTree a = Empty
              | Node Color (RBTree a) a (RBTree a)

insert::(Ord a)=>RBTree a -> a -> RBTree a
insert t x = makeBlack $ ins t where
    ins Empty = Node R Empty x Empty
    ins (Node color l k r)
        | x < k     = balance color (ins l) k r
        | otherwise = balance color l k (ins r)
    makeBlack(Node _ l k r) = Node B l k r

balance::Color -> RBTree a -> a -> RBTree a -> RBTree a
balance B (Node R (Node R a x b) y c) z d =
        Node R (Node B a x b) y (Node B c z d)
balance B (Node R a x (Node R b y c)) z d =
        Node R (Node B a x b) y (Node B c z d)
balance B a x (Node R b y (Node R c z d)) =
        Node R (Node B a x b) y (Node B c z d)
balance B a x (Node R (Node R b y c) z d) =
        Node R (Node B a x b) y (Node B c z d)
balance color l k r = Node color l k r

\end{lstlisting}

\subsection{删除}

使用“双重黑色”概念实现的红黑树删除算法。

\begin{lstlisting}
data Color = R | B | BB deriving (Show, Eq) -- BB为双重黑色
data RBTree a = Empty
              | Node Color (RBTree a) a (RBTree a)
              | BBEmpty -- 双重黑色空节点

min::RBTree a -> a
min (Node _ Empty x _) = x
min (Node _ l _ _) = min l

isEmpty :: (RBTree a) -> Bool
isEmpty Empty = True
isEmpty _ = False

delete::(Ord a)=>RBTree a -> a -> RBTree a
delete t x = blackenRoot (del t x) where
    del Empty _ = Empty
    del (Node color l k r) x
        | x < k = fixDB color (del l x) k r
        | x > k = fixDB color l k (del r x)
        -- x == k, 删除此节点
        | isEmpty l = if color==B then makeBlack r else r
        | isEmpty r = if color==B then makeBlack l else l
        | otherwise = fixDB color l k' (del r k') where k'= min r
    blackenRoot (Node _ l k r) = Node B l k r
    blackenRoot _ = Empty

makeBlack::RBTree a -> RBTree a
makeBlack (Node B l k r) = Node BB l k r -- 双重黑色
makeBlack (Node _ l k r) = Node B l k r
makeBlack Empty = BBEmpty
makeBlack t = t

fixDB::Color -> RBTree a -> a -> RBTree a -> RBTree a
-- 兄弟节点为黑色,有一个红色的侄子
fixDB color a@(Node BB _ _ _) x (Node B (Node R b y c) z d) =
      Node color (Node B (makeBlack a) x b) y (Node B c z d)
fixDB color BBEmpty x (Node B (Node R b y c) z d) =
      Node color (Node B Empty x b) y (Node B c z d)
fixDB color a@(Node BB _ _ _) x (Node B b y (Node R c z d)) =
      Node color (Node B (makeBlack a) x b) y (Node B c z d)
fixDB color BBEmpty x (Node B b y (Node R c z d)) =
      Node color (Node B Empty x b) y (Node B c z d)
fixDB color (Node B a x (Node R b y c)) z d@(Node BB _ _ _) =
      Node color (Node B a x b) y (Node B c z (makeBlack d))
fixDB color (Node B a x (Node R b y c)) z BBEmpty =
      Node color (Node B a x b) y (Node B c z Empty)
fixDB color (Node B (Node R a x b) y c) z d@(Node BB _ _ _) =
      Node color (Node B a x b) y (Node B c z (makeBlack d))
fixDB color (Node B (Node R a x b) y c) z BBEmpty =
      Node color (Node B a x b) y (Node B c z Empty)
-- 兄弟为红色
fixDB B a@(Node BB _ _ _) x (Node R b y c) = fixDB B (fixDB R a x b) y c
fixDB B a@BBEmpty x (Node R b y c) = fixDB B (fixDB R a x b) y c
fixDB B (Node R a x b) y c@(Node BB _ _ _) = fixDB B a x (fixDB R b y c)
fixDB B (Node R a x b) y c@BBEmpty = fixDB B a x (fixDB R b y c)
-- 兄弟和两个侄子都是黑色。
fixDB color a@(Node BB _ _ _) x (Node B b y c) =
      makeBlack (Node color (makeBlack a) x (Node R b y c))
fixDB color BBEmpty x (Node B b y c) =
      makeBlack (Node color Empty x (Node R b y c))
fixDB color (Node B a x b) y c@(Node BB _ _ _) = m
      akeBlack (Node color (Node R a x b) y (makeBlack c))
fixDB color (Node B a x b) y BBEmpty =
      makeBlack (Node color (Node R a x b) y Empty)
-- 其他情况
fixDB color l k r = Node color l k r
\end{lstlisting}

\section{基数树}

\subsection{整数Trie}
小端整数Trie.

\lstset{language=Haskell}
\begin{lstlisting}[caption=整数Trie的代数数据类型定义]
data IntTrie a = Empty
               | Branch (IntTrie a) (Maybe a) (IntTrie a)
\end{lstlisting}

\lstset{language=Haskell}
\begin{lstlisting}[caption=整数Trie的插入算法]
insert t 0 x = Branch (left t) (Just x) (right t)
insert t k x
    | even k = Branch (insert (left t) (k `div` 2) x) (value t) (right t)
    | otherwise = Branch (left t) (value t) (insert (right t) (k `div` 2) x)

left (Branch l _ _) = l
left Empty = Empty

right (Branch _ _ r) = r
right Empty = Empty

value (Branch _ v _) = v
value Empty = Nothing
\end{lstlisting}

\lstset{language=Haskell}
\begin{lstlisting}[caption=整数Trie的查找算法]
search Empty k = Nothing
search t 0 = value t
search t k = if even k then search (left t) (k `div` 2)
             else search (right t) (k `div` 2)
\end{lstlisting}

\subsection{整数前缀树}

\lstset{language=Haskell}
\begin{lstlisting}[caption=整数前缀树的定义]
type Key = Int
type Prefix = Int
type Mask = Int

data IntTree a = Empty
               | Leaf Key a
               | Branch Prefix Mask (IntTree a) (IntTree a)
\end{lstlisting}

\lstset{language=Haskell}
\begin{lstlisting}[caption=整数前缀树的插入]
import Data.Bits

insert t k x
   = case t of
       Empty -> Leaf k x
       Leaf k' x' -> if k==k' then Leaf k x
                     else join k (Leaf k x) k' t -- t@(Leaf k' x')
       Branch p m l r
          | match k p m -> if zero k m
                           then Branch p m (insert l k x) r
                           else Branch p m l (insert r k x)
          | otherwise -> join k (Leaf k x) p t -- t@(Branch p m l r)

join p1 t1 p2 t2 = if zero p1 m then Branch p m t1 t2
                                else Branch p m t2 t1
    where
      (p, m) = lcp p1 p2

lcp :: Prefix -> Prefix -> (Prefix, Mask)
lcp p1 p2 = (p, m) where
    m = bit (highestBit (p1 `xor` p2))
    p = mask p1 m

highestBit x = if x == 0 then 0 else 1 + highestBit (shiftR x 1)

mask x m = (x .&. complement (m-1)) -- complement means bit-wise not.

zero x m = x .&. (shiftR m 1) == 0

match k p m = (mask k m) == p
\end{lstlisting}

\lstset{language=Haskell}
\begin{lstlisting}[caption=整数前缀树的查找]
search t k
  = case t of
      Empty -> Nothing
      Leaf k' x -> if k == k' then Just x else Nothing
      Branch p m l r
             | match k p m -> if zero k m then search l k
                              else search r k
             | otherwise -> Nothing
\end{lstlisting}

\subsection{字符Trie}

\lstset{language=Haskell}
\begin{lstlisting}[caption=字符Trie的定义]
data Trie a = Trie { value :: Maybe a
                   , children :: [(Char, Trie a)]}

empty = Trie Nothing []
\end{lstlisting}

\lstset{language=Haskell}
\begin{lstlisting}[caption=字符Trie的插入]
insert t []     x = Trie (Just x)  (children t)
insert t (k:ks) x = Trie (value t) (ins (children t) k ks x) where
    ins [] k ks x = [(k, (insert empty ks x))]
    ins (p:ps) k ks x = if fst p == k
                        then (k, insert (snd p) ks x):ps
                        else p:(ins ps k ks x)
\end{lstlisting}

\lstset{language=Haskell}
\begin{lstlisting}[caption=字符Trie的查找]
find t [] = value t
find t (k:ks) = case lookup k (children t) of
                  Nothing -> Nothing
                  Just t' -> find t' ks
\end{lstlisting}

\subsection{字符前缀树}

\lstset{language=Haskell}
\begin{lstlisting}[caption=字符前缀树的定义]
data PrefixTree k v = PrefixTree { value :: Maybe v
                                 , children :: [([k], PrefixTree k v)]}

empty = PrefixTree Nothing []

leaf x = PrefixTree (Just x) []
\end{lstlisting}


\lstset{language=Haskell}
\begin{lstlisting}[caption=前缀树的插入]
import Data.List (isPrefixOf)

insert :: Eq k => PrefixTree k v -> [k] -> v -> PrefixTree k v
insert t ks x = PrefixTree (value t) (ins (children t) ks x) where
    ins []     ks x = [(ks, leaf x)]
    ins (p@(ks', t') : ps) ks x
        | ks' == ks
            = (ks, PrefixTree (Just x) (children t')) : ps  -- overwrite
        | match ks' ks
            = (branch ks x ks' t') : ps
        | otherwise
            = p : (ins ps ks x)

match x y = x /= [] && y /= [] && head x == head y

branch :: Eq k => [k] -> v -> [k] -> PrefixTree k v -> ([k], PrefixTree k v)
branch ks1 x ks2 t2
    | ks1 == ks
        -- ex: insert "an" into "another"
        = (ks, PrefixTree (Just x) [(ks2', t2)])
    | ks2 == ks
        -- ex: insert "another" into "an"
        = (ks, insert t2 ks1' x)
    | otherwise = (ks, PrefixTree Nothing [(ks1', leaf x), (ks2', t2)])
   where
      ks = lcp ks1 ks2
      m = length ks
      ks1' = drop m ks1
      ks2' = drop m ks2

-- longest common prefix
lcp :: Eq k => [k] -> [k] -> [k]
lcp [] _ = []
lcp _ [] = []
lcp (x:xs) (y:ys) = if x==y then x : (lcp xs ys) else []
\end{lstlisting}

\lstset{language=Haskell}
\begin{lstlisting}[caption=字符前缀树的查找]
find :: Eq k => PrefixTree k v -> [k] -> Maybe v
find t = find' (children t) where
    find' [] _ = Nothing
    find' (p@(ks', t') : ps) ks
          | ks' == ks = value t'
          | ks' `isPrefixOf` ks = find t' (diff ks ks')
          | otherwise = find' ps ks
    diff ks1 ks2 = drop (length (lcp ks1 ks2)) ks1
\end{lstlisting}

\lstset{language=Haskell}
\begin{lstlisting}[caption=在前缀树中查找最多$n$个结果]
import Control.Arrow (first)

get n t k = take n $ findAll t k

findAll :: Eq k => PrefixTree k v -> [k] -> [([k], v)]
findAll (PrefixTree Nothing cs) [] = enum cs
findAll (PrefixTree (Just x) cs) [] = ([], x) : enum cs
findAll (PrefixTree _ cs) k = find' cs k
  where
    find' [] _ = []
    find' ((k', t') : ps) k
          | k `isPrefixOf` k'
              = map (first (k' ++)) (findAll t' [])
          | k' `isPrefixOf` k
              = map (first (k' ++)) (findAll t' $ drop (length k') k)
          | otherwise = find' ps k

enum :: Eq k => [([k], PrefixTree k v)] -> [([k], v)]
enum = concatMap (\(k, t) -> map (first (k ++)) (findAll t []))
\end{lstlisting}

\lstset{language=Haskell}
\begin{lstlisting}[caption=T9查找]
import qualified Data.Map as Map

mapT9 = Map.fromList [('1', ",."), ('2', "abc"), ('3', "def"), ('4', "ghi"),
                      ('5', "jkl"), ('6', "mno"), ('7', "pqrs"), ('8', "tuv"),
                      ('9', "wxyz")]

rmapT9 = Map.fromList $ concatMap (\(d, s) -> [(c, d) | c <- s]) $ Map.toList mapT9

digits = map (\c -> Map.findWithDefault '#' c rmapT9)

findT9 :: PrefixTree Char v -> String -> [String]
findT9 t [] = [""]
findT9 t k = concatMap find prefixes
  where
    n = length k
    find (s, t') = map (take n . (s++)) $ findT9 t' (k `diff` s)
    diff x y = drop (length y) x
    prefixes = [(s, t') | (s, t') <- children t, let ds = digits s in
                          ds `isPrefixOf` k || k `isPrefixOf` ds]
\end{lstlisting}

\ifx\wholebook\relax \else
\end{document}
\fi
