\ifx\wholebook\relax \else
% ------------------------

\documentclass[b5paper]{article}

\usepackage[en]{../../prelude}

\setcounter{page}{1}

\begin{document}

\title{Preface}

\author{Larry~LIU~Xinyu
\thanks{{\bfseries Larry LIU Xinyu } \newline
  Email: liuxinyu95@gmail.com \newline}
  }

\maketitle
\fi

\markboth{Preface}{Elementary Algorithms}

Programmers learn elementary algorithms at school. Except for programming contest, code interview, they seldom use algorithms in commercial software development. When talking about algorithms in AI and machine learning, it actually means scientific modeling, but not about data structure or elementary algorithm. Even when programmers need them, they have already been provided in libraries. It seems quite enough to know about how to use the library as a tool but not `re-invent the wheel'.

I would say elementary algorithms are critical in solving `interesting problems', the usefulness of the problem set aside. Let's start with two problems.

\section{The smallest free number}
\label{min-free} \index{minimum free number}

Richard Bird gives an interesting programming problem to find the minimum number that not appears in a given list(Chapter 1, \cite{Bird-book}). It's common to use a number as the identifier (Id) to index entities. At any time, a number is either occupied or free. When client tries to acquire a new number as index, we want to always allocate the smallest available one. Suppose numbers are non-negative integers and those being occupied are recorded in a list, for example:

\begin{verbatim}
[18, 4, 8, 9, 16, 1, 14, 7, 19, 3, 0, 5, 2, 11, 6]
\end{verbatim}

How can we find the smallest free number, which is 10, from the list? It seems quite easy to figure out the solution.

\begin{algorithmic}[1]
\Function{Min-Free}{$A$}
  \State $x \gets 0$
  \Loop
    \If{$x \notin A$}
      \State \Return $x$
    \Else
      \State $x \gets x + 1$
    \EndIf
  \EndLoop
\EndFunction
\end{algorithmic}

Where the $\notin$ is realized like below.

\begin{algorithmic}[1]
\Function{`$\notin$'}{$x, X$}
  \For{$i \gets 1 $ to $|X|$}
    \If{$x = X[i]$}
      \State \Return False
    \EndIf
  \EndFor
  \State \Return True
\EndFunction
\end{algorithmic}

Some environments have built-in implementation to test if an element is in a list. Below is an example program.

\lstset{language=Python, frame=single}
\begin{lstlisting}
def minfree(lst):
    i = 0
    while True:
        if i not in lst:
            return i
        i = i + 1
\end{lstlisting}

However, when there are millions of numbers being used, this solution performs poor. The time spent is quadratic to the length of the list. In a computer with 2 cores of 2.10 GHz CPU, and 2G RAM, the C implementation takes 5.4s to search the minimum free number among 100,000 numbers, and takes more than 8 minutes to handle a million numbers.

\subsection{Improvement}
The key idea to improve the solution is based on the fact that, for $n$ numbers $x_1, x_2, ..., x_n$, if there exists free number, some $x_i$ must be outside the range $[0, n)$; otherwise the list is exactly some permutation of $0, 1, ..., n - 1$ hence $n$ should be returned as the minimum free number. In summary:

\be
\textit{minfree}(x_1, x_2, ..., x_n) \leq n
\label{min-free}
\ee

A better solution is to use an array of $n + 1$ flags to mark whether a number in range $[0, n]$ is free.

\begin{algorithmic}[1]
\Function{Min-Free}{$A$}
  \State $F \gets $[False, False, ..., False] where $|F| = n+1$
  \For{$\forall x \in A$}
    \If{$x < n$}
      \State $F[x] \gets$ True
    \EndIf
  \EndFor
  \For{$i \gets [0, n]$}
    \If{$F[i] =$ False}
      \State \Return $i$
    \EndIf
  \EndFor
\EndFunction
\end{algorithmic}

Line 2 initializes a flag array all of False values. Then we scan all numbers in $A$ and mark the corresponding flag to True if the value is less than $n$. Finally, we iterate to find the first False flag. This program takes time proportion to $n$. It uses $n + 1$ flags to cover the special case that $sorted(A) = [0, 1, 2, ..., n-1]$. This solution is much faster than the brute force one. In the same computer, the Python implementation takes 0.02s when dealing with 100,000 numbers.

Although this solution only takes $O(n)$ time, it needs additional $O(n)$ space to store the flags. We haven't tuned it yet. Each time the program allocates memory to create an array of $n + 1$ flags, then releases it when finish. Such memory allocation and release is expensive and cost a lot of processing time.

To improve it, we can allocate the memory in advance for later reusing, and change to bit-wise flags instead of array. For example as the following C program:

\begin{lstlisting}[language=C]
#define N 1000000
#define WORD_LENGTH (sizeof(int) * 8)

void setbit(unsigned int* bits, unsigned int i) {
    bits[i / WORD_LENGTH] |= 1 << (i % WORD_LENGTH);
}

int testbit(unsigned int* bits, unsigned int i) {
    return bits[i / WORD_LENGTH] & (1 << (i % WORD_LENGTH));
}

unsigned int bits[N / WORD_LENGTH + 1];

int minfree(int* xs, int n) {
  int i, len = N/WORD_LENGTH + 1;
  for (i = 0; i < len; ++i) {
      bits[i]=0;
  }
  for (i=0; i < n; ++i) {
      if(xs[i] < n) {
          setbit(bits, xs[i]);
      }
  }
  for (i=0; i <= n; ++i) {
      if (!testbit(bits, i)) {
          return i;
      }
  }
}
\end{lstlisting}

This program can handle 1 million numbers in 0.023s in the same computer.

\subsection{Divide and Conquer}
The above improvement costs $O(n)$ additional space for flags, can we eliminate it? The divide and conquer strategy is to break the problem into smaller ones, then solve them separately to get the answer.

We can put numbers $x_i \leq \lfloor n/2 \rfloor$ into a sub-list $A'$ and put the rest into another sub-list $A''$. According to (\ref{min-free}), if the length of $A'$ equals to $\lfloor n/2 \rfloor$, it means $A'$ is `full'. The minimum free number must be in $A''$. We can recursively search in $A''$ which is shorter the original list. Otherwise, it means the minimum free number is in $A'$, which again leads to a smaller problem.

When search in $A''$, the conditions change a bit. We do not start from $0$, but from $\lfloor n/2 \rfloor + 1$ as the new lower bound. We define the algorithm as $search(A, l, u)$, where $l$ is
the lower bound and $u$ is the upper bound index. For the empty list as a special case, we return the $l$ as the result.

\[
minfree(A) = search(A, 0, |A|-1)
\]

\[
\begin{array}{rcl}
search(\nil, l, u) & = & l \\
search(A, l, u) & = & \begin{cases}
       |A'| = m - l + 1 & search(A'', m+1, u) \\
       otherwise & search(A',  l, m) \\
\end{cases}
\end{array}
\]

where

\[ \begin{array}{rcl}
m & = & \displaystyle \lfloor \frac{l+u}{2} \rfloor \\
A' & = & \{ x | x \in A, x \leq m \} \\
A''& = & \{ x | x \in A, x > m \} \\
\end{array} \]

This algorithm doesn't need additional space\footnote{The recursion takes $O(\lg n)$ stack spaces, but it can be eliminated through tail recursion optimization}. Each recursive call performs $O(|A|)$ comparisons to build $A'$ and $A''$. After that the problem scale halves. Therefore, the time is bound to $T(n) = T(n/2) + O(n)$, which reduce to $O(n)$ according to master theorem. Alternatively, observe that the first call takes $O(n)$ to build $A'$ and $A''$ and the second call takes $O(n/2)$, and $O(n/4)$ for the third... The total time is $O(n + n/2 + n/4 + ...) = O(2n) = O(n)$.

Below example Haskell program implements this algorithm.

\begin{Haskell}
minFree xs = bsearch xs 0 (length xs - 1)

bsearch xs l u | xs == [] = l
               | length as == m - l + 1 = bsearch bs (m+1) u
               | otherwise = bsearch as l m
    where
      m = (l + u) `div` 2
      (as, bs) = partition (<=m) xs
\end{Haskell}

\subsection{Expressiveness and performance}
One may concern the performance of this divide and conquer algorithm. There are $O(\lg n)$ recursive calls, which need additional stack space. If wanted, we can eliminate the recursion:

\begin{algorithmic}[1]
\Function{Min-Free}{$A$}
  \State $l \gets 0, u \gets |A|$
  \While{$u - l > 0$}
    \State $m \gets l + \dfrac{u - l}{2}$
    \State $left \gets l$
    \For{$right \gets l$ to $u - 1$}
      \If{$A[right] \leq m$}
        \State $A[left] \leftrightarrow A[right]$
        \State $left \gets left + 1$
      \EndIf
    \EndFor
    \If{$left < m + 1$}
      \State $u \gets left$
    \Else
      \State $l \gets left$
    \EndIf
  \EndWhile
\EndFunction
\end{algorithmic}

As shown in figure \ref{fig:divide}, this program re-arranges the array such that all elements before $left$ are less than or equal to $m$; while those between $left$ and $right$ are greater than $m$.

\begin{figure}[htbp]
  \centering
  \includegraphics[scale=0.7]{img/divide-by-m.ps}
  \caption{Divide the array, all $A[i] \leq m$ where $0 \leq i < left$; while all $A[i] > m$ where $left \leq i < right$. The rest elements haven't been processed yet.} \label{fig:divide}
\end{figure}

This solution is fast and needn't extra stack space. However, compare to the previous recursive one, there is some expressiveness drops. Depends on individual taste, one may prefer one over the other.

\section{The number puzzle, power of data structure}

If the first problem, to find the minimum free number, is a some what
useful in practice, this problem is a `pure' one for fun. The puzzle
is to find the 1,500th number, which only contains factor 2, 3 or 5.
The first 3 numbers are of course 2, 3, and 5. Number $60 = 2^23^15^1$,
However it is the 25th number. Number $21 = 2^03^17^1$, isn't a valid
number because it contains a factor 7. The first 10 such numbers are list
as the following.

2,3,4,5,6,8,9,10,12,15

If we consider $1=2^03^05^0$, then 1 is also a valid number and it is
the first one.

\subsection{The brute-force solution}
It seems the solution is quite easy without need any serious algorithms.
We can check all numbers from 1, then extract all factors of 2, 3 and 5
to see if the left part is 1.

\begin{algorithmic}[1]
\Function{Get-Number}{$n$}
  \State $x \gets 1$
  \State $i \gets 0$
  \Loop
    \If{\Call{Valid?}{$x$}}
      \State $i \gets i + 1$
      \If{$i = n$}
        \State \Return $x$
      \EndIf
    \EndIf
    \State $x \gets x + 1$
  \EndLoop
\EndFunction
\Statex
\Function{Valid?}{$x$}
  \While{$x \bmod 2 = 0$}
    \State $x \gets x / 2$
  \EndWhile
  \While{$x \bmod 3 = 0$}
    \State $x \gets x / 3$
  \EndWhile
  \While{$x \bmod 5 = 0$}
    \State $x \gets x / 5$
  \EndWhile
  \If{$x = 1$}
    \State \Return $True$
  \Else
    \State \Return $False$
  \EndIf
\EndFunction
\end{algorithmic}

This `brute-force' algorithm works for most small $n$. However, to find
the 1500th number (which is 859963392), the C program based on this
algorithm takes 40.39 seconds in my computer. I have to kill the program
after 10 minutes when I increased $n$ to 15,000.

\subsection{Improvement 1}
Analysis of the above algorithm shows that modular and divide calculations
are very expensive \cite{Bentley}. And they executed a lot in loops.
Instead of checking a number contains only 2, 3, or 5 as factors, one
alternative solution is to construct such number by these factors.

We start from 1, and times it with 2, or 3, or 5 to generate rest numbers.
The problem turns to be how to generate the candidate number in order?
One handy way is to utilize the queue data structure.

A queue data structure is used to push elements at one end, and pops
them at the other end. So that the element be pushed first is also
be popped out first. This property is called FIFO (First-In-First-Out).

The idea is to push 1 as the only element to the queue, then we pop
an element, times it with 2, 3, and 5, to get 3 new elements. We
then push them back to the queue in order. Note that, the new elements may
have already existed in the queue. In such case, we just drop the
element. The new element may also smaller than the others in the queue,
so we must put them to the correct position. Figure \ref{fig:queues}
illustrates this idea.

\begin{figure}[htbp]
       \begin{center}
       	  \includegraphics[scale=0.5]{img/q1.ps}
       	  \includegraphics[scale=0.5]{img/q2.ps}
       	  \includegraphics[scale=0.5]{img/q3.ps}
       	  \includegraphics[scale=0.5]{img/q4.ps}
        \caption{First 4 steps of constructing numbers with a queue. \newline
        1. Queue is initialized with 1 as the only element;\newline
        2. New elements 2, 3, and 5 are pushed back; \newline
        3. New elements 4, 6, and 10, are pushed back in order; \newline
        4. New elements 9 and 15 are pushed back, element 6 already exists.} \label{fig:queues}
       \end{center}
\end{figure}

This algorithm is shown as the following.

\begin{algorithmic}[1]
\Function{Get-Number}{$n$}
  \State $Q \gets NIL$
  \State \Call{Enqueue}{$Q, 1$}
  \While{$n > 0$}
    \State $x \gets$ \Call{Dequeue}{$Q$}
    \State \Call{Unique-Enqueue}{$Q, 2x$}
    \State \Call{Unique-Enqueue}{$Q, 3x$}
    \State \Call{Unique-Enqueue}{$Q, 5x$}
    \State $n \gets n-1$
  \EndWhile
  \State \Return $x$
\EndFunction
\Statex
\Function{Unique-Enqueue}{$Q, x$}
  \State $i \gets 0$
  \While{$i < |Q| \wedge Q[i] < x$}
    \State $i \gets i + 1$
  \EndWhile
  \If{$i < |Q| \wedge x = Q[i]$}
    \State \Return
  \EndIf
  \State \Call{Insert}{$Q, i, x$}
\EndFunction
\end{algorithmic}

The insert function takes $O(|Q|)$ time to find the proper position and insert
it. If the element has already existed, it just returns.

A rough estimation tells that the length of the queue increase proportion to $n$,
(Each time, we extract one element, and pushed 3 new, the increase ratio $\leq$ 2),
so the total running time is $O(1+2+3+...+n) = O(n^2)$.

Figure\ref{fig:big-O-1q} shows the number of queue access time against $n$.
It is quadratic curve which reflect the $O(n^2)$ performance.

\begin{figure}[htbp]
       \begin{center}
       	  \includegraphics[scale=0.5]{img/big-O-1q.eps}
        \caption{Queue access count v.s. $n$.} \label{fig:big-O-1q}
       \end{center}
\end{figure}

The C program based on this algorithm takes only 0.016[s] to get the right answer
859963392. Which is 2500 times faster than the brute force solution.

%% Functional 1Q solution
Improvement 1 can also be considered in recursive way. Suppose $X$ is the infinity
series for all numbers which only contain factors of 2, 3, or 5. The following
formula shows an interesting relationship.

\be
  X = \{1\} \cup \{2x: \forall x \in X\} \cup \{3x: \forall x \in X \} \cup \{5x: \forall x \in X \}
\ee

Where we can define $\cup$ to a special form so that all elements are stored in order
as well as unique to each other. Suppose that $X=\{x_1, x_2, x_3...\}$, $Y=\{y_1, y_2, y_3, ...\}$, $X' = \{x_2, x_3, ...\}$ and $Y'=\{y_2, y_3, ...\}$. We have

\[
X \cup Y = \left \{
  \begin{array}{r@{\quad:\quad}l}
  X & Y = \phi \\
  Y & X = \phi \\
  \{ x_1, X' \cup Y \} & x_1 < y_1 \\
  \{ x_1, X' \cup Y' \} & x_1 = y_1 \\
  \{ y_1, X \cup Y' \} & x_1 > y_1
  \end{array}
\right.
\]

In a functional programming language such as Haskell, which supports
lazy evaluation, The above infinity series functions can be translate
into the following program.

\lstset{language=Haskell}
\begin{lstlisting}
ns = 1:merge (map (*2) ns) (merge (map (*3) ns) (map (*5) ns))

merge [] l = l
merge l [] = l
merge (x:xs) (y:ys) | x <y = x : merge xs (y:ys)
                    | x ==y = x : merge xs ys
                    | otherwise = y : merge (x:xs) ys
\end{lstlisting}

By evaluate ns !! (n-1), we can get the 1500th number as
below.

\begin{verbatim}
>ns !! (1500-1)
859963392
\end{verbatim}

\subsection{Improvement 2}
Considering the above solution, although it is much faster than the brute-force one,
It still has some drawbacks. It produces many duplicated numbers and they are
finally dropped when examine the queue. Secondly, it does linear scan and insertion
to keep the order of all elements in the queue, which degrade the ENQUEUE operation
from $O(1)$ to $O(|Q|)$.

If we use three queues instead of using only one, we can improve the solution one
step ahead. Denote these queues as $Q_2$, $Q_3$, and $Q_5$, and we initialize
them as $Q_2=\{ 2 \}$, $Q_3 = \{ 3\}$ and $Q_5 = \{ 5 \}$. Each time we DEQUEUEed
the smallest one from $Q_2$, $Q_3$, and $Q_5$ as $x$. And do the following test:

\begin{itemize}
\item If $x$ comes from $Q_2$, we ENQUEUE $2x$, $3x$, and $5x$ back to
$Q_2$, $Q_3$, and $Q_5$ respectively;
\item If $x$ comes from $Q_3$, we only need ENQUEUE $3x$ to $Q_3$, and $5x$ to $Q_5$;
We needn't ENQUEUE $2x$ to $Q_2$, because $2x$ have already existed in $Q_3$;
\item If $x$ comes from $Q_5$, we only need ENQUEUE $5x$ to $Q_5$; there is
no need to ENQUEUE $2x$, $3x$ to $Q_2$, $Q_3$ because they have already been
in the queues;
\end{itemize}

We repeatedly ENQUEUE the smallest one until we find the $n$-th element.

\begin{figure}[htbp]
       \begin{center}
       	  \includegraphics[scale=0.5]{img/q235-1.ps}
       	  \includegraphics[scale=0.5]{img/q235-2.ps}
       	  \includegraphics[scale=0.5]{img/q235-3.ps}
       	  \includegraphics[scale=0.5]{img/q235-4.ps}
        \caption{First 4 steps of constructing numbers with $Q_2$, $Q_3$, and $Q_5$. \newline
        1. Queues are initialized with 2, 3, 5 as the only element;\newline
        2. New elements 4, 6, and 10 are pushed back; \newline
        3. New elements 9, and 15, are pushed back; \newline
        4. New elements 8, 12, and 20 are pushed back; \newline
        5. New element 25 is pushed back.} \label{fig:q235}
       \end{center}
\end{figure}

The algorithm based on this idea is implemented as below.

\begin{algorithmic}[1]
\Function{Get-Number}{$n$}
  \If{$n = 1$}
    \State \Return $1$
  \Else
    \State $Q_2 \gets \{ 2 \}$
    \State $Q_3 \gets \{ 3 \}$
    \State $Q_5 \gets \{ 5 \}$
    \While{$n > 1$}
      \State $x \gets min($\Call{Head}{$Q_2$}, \Call{Head}{$Q_3$}, \Call{Head}{$Q_5$}$)$
      \If{$x = $ \Call{Head}{$Q_2$}}
        \State \Call{Dequeue}{$Q_2$}
        \State \Call{Enqueue}{$Q_2, 2x$}
        \State \Call{Enqueue}{$Q_3, 3x$}
        \State \Call{Enqueue}{$Q_5, 5x$}
      \ElsIf{$x=$ \Call{Head}{$Q_3$}}
        \State \Call{Dequeue}{$Q_3$}
        \State \Call{Enqueue}{$Q_3, 3x$}
        \State \Call{Enqueue}{$Q_5, 5x$}
      \Else
        \State \Call{Dequeue}{$Q_5$}
        \State \Call{Enqueue}{$Q_5, 5x$}
      \EndIf
      \State $n \gets n - 1$
    \EndWhile
    \State \Return $x$
  \EndIf
\EndFunction
\end{algorithmic}

This algorithm loops $n$ times, and within each loop, it extract one head
element from the three queues, which takes constant time. Then it appends
one to three new elements at the end of queues which bounds to constant time
too. So the total time of the algorithm bounds to $O(n)$. The C++ program
translated from this algorithm shown below takes less than 1 $\mu$s to
produce the 1500th number, 859963392.

\lstset{language=C++}
\begin{lstlisting}
typedef unsigned long Integer;

Integer get_number(int n){
  if(n==1)
    return 1;
  queue<Integer> Q2, Q3, Q5;
  Q2.push(2);
  Q3.push(3);
  Q5.push(5);
  Integer x;
  while(n-- > 1){
    x = min(min(Q2.front(), Q3.front()), Q5.front());
    if(x==Q2.front()){
      Q2.pop();
      Q2.push(x*2);
      Q3.push(x*3);
      Q5.push(x*5);
    }
    else if(x==Q3.front()){
      Q3.pop();
      Q3.push(x*3);
      Q5.push(x*5);
    }
    else{
      Q5.pop();
      Q5.push(x*5);
    }
  }
  return x;
}
\end{lstlisting}

This solution can be also implemented in Functional way. We define
a function $take(n)$, which will return the first $n$ numbers contains
only factor 2, 3, or 5.

\[
  take(n) = f(n, \{1\}, \{2\}, \{3\}, \{5\})
\]
Where
\[
 f(n, X, Q_2, Q_3, Q_5) = \left \{
  \begin{array}{r@{\quad:\quad}l}
  X & n = 1 \\
  f(n-1, X \cup \{x\}, Q_2', Q_3', Q_5') & otherwise
  \end{array}
\right.
\]

\[
 x = min(Q_{21}, Q_{31}, Q_{51})
\]
\[
 Q_2', Q_3', Q_5' = \left \{
 \begin{array}{r@{\quad:\quad}l}
 \{Q_{22}, Q_{23}, ...\} \cup \{2x\}, Q_3 \cup \{3x\}, Q_5 \cup \{5x\} & x = Q_{21} \\
 Q_2, \{Q_{32}, Q_{33}, ...\} \cup \{3x\}, Q5 \cup \{5x\} & x = Q_{31} \\
 Q_2, Q_3, \{Q_{52}, Q_{53}, ...\} \cup \{5x\} & x = Q_{51}
 \end{array}
 \right.
\]

And these functional definition can be realized in Haskell as the following.

\lstset{language=Haskell}
\begin{lstlisting}
ks 1 xs _ = xs
ks n xs (q2, q3, q5) = ks (n-1) (xs++[x]) update
    where
      x = minimum $ map head [q2, q3, q5]
      update | x == head q2 = ((tail q2)++[x*2], q3++[x*3], q5++[x*5])
             | x == head q3 = (q2, (tail q3)++[x*3], q5++[x*5])
             | otherwise = (q2, q3, (tail q5)++[x*5])

takeN n = ks n [1] ([2], [3], [5])
\end{lstlisting} %$

Invoke `last takeN 1500' will generate the correct answer 859963392.

% ================================================================
%                 Short summary
% ================================================================
\section{Notes and short summary}
If review the 2 puzzles, we found in both cases, the brute-force solutions
are so weak. In the first problem, it's quite poor in dealing with
long ID list, while in the second problem, it doesn't work at all.

The first problem shows the power of algorithms, while the second
problem tells why data structure is important. There are plenty
of interesting problems, which are hard to solve before computer
was invented. With the aid of computer and programming, we are able
to find the answer in a quite different way. Compare to what we
learned in mathematics course in school, we haven't been taught the method
like this.

While there have been already a lot of wonderful books about
algorithms, data structures and math, however, few of them
provide the comparison between the procedural solution and
the functional solution. From the above discussion, it can be
found that functional solution sometimes is very expressive
and they are close to what we are familiar in mathematics.

This series of post focus on providing both imperative and functional
algorithms and data structures. Many functional data structures
can be referenced from Okasaki's book\cite{okasaki-book}. While
the imperative ones can be founded in classic text books \cite{CLRS}
or even in WIKIpedia.
Multiple
programming languages, including, C, C++, Python, Haskell, and
Scheme/Lisp will be used. In order to make it easy to read
by programmers with different background, pseudo code and mathematical
function are the regular descriptions of each post.

The author is NOT a native English speaker, the reason why
this book is only available in English for the time being
is because the contents are still changing frequently. Any
feedback, comments, or criticizes are welcome.

\section{Structure of the contents}
In the following series of post, I'll first introduce about
elementary data structures before algorithms, because many
algorithms need knowledge of data structures as prerequisite.

The `hello world' data structure, binary search tree is the
first topic; Then we introduce how to solve the balance problem
of binary search tree. After that, I'll show other interesting
trees. Trie, and Prefix trees are useful in text manipulation.
While B-trees are commonly used in file system and data base
implementation.

The second part of data structures is about heaps. We'll
provide a general Heap definition and introduce about binary
heaps by array and by explicit binary trees. Then we'll
extend to K-ary heaps including Binomial heaps, Fibonacci
heaps, and pairing heaps.

Array and queues are considered among the easiest data structures
typically, However, we'll show how difficult to implement
them in the third part.

As the elementary sort algorithms, we'll introduce insertion
sort, quick sort, merge sort etc in both imperative way
and functional way.

The final part is about searching, besides the element
searching, we'll also show string matching algorithms
such as KMP.

\begin{thebibliography}{99}

\bibitem{Bird-book}
Richard Bird. ``Pearls of functional algorithm design''. Cambridge University Press; 1 edition (November 1, 2010). ISBN-10: 0521513383

\bibitem{Bentley}
Jon Bentley. ``Programming Pearls(2nd Edition)''. Addison-Wesley Professional; 2 edition (October 7, 1999). ISBN-13: 978-0201657883

\bibitem{okasaki-book}
Chris Okasaki. ``Purely Functional Data Structures''. Cambridge university press, (July 1, 1999), ISBN-13: 978-0521663502

\bibitem{CLRS}
Thomas H. Cormen, Charles E. Leiserson, Ronald L. Rivest and Clifford Stein. ``Introduction to Algorithms, Second Edition''. The MIT Press, 2001. ISBN: 0262032937.

\end{thebibliography}

\ifx\wholebook\relax \else
\end{document}
\fi
